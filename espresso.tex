\documentclass{article}
\usepackage{pgfplots}
\usetikzlibrary{decorations.pathmorphing,intersections,fillbetween}
\pgfplotsset{compat=1.10}

\begin{document}

\small
\begin{tabular}{r|p{3.6em}|p{3.6em}|p{3.6em}|p{3.6em}|p{3.6em}|p{3.6em}|p{3.6em}|p{3.6em}|}
              & Espresso & Steamed Milk & Milk Foam & Brewed Coffee & Water & Cream & Steamed Half-and-half & Choc. Syrup\\ \hline
  Espresso    &    .4    &         -    &       -   &           -   &    -  &    -  &                -      &       -     \\
  Caffe Latte &    .4    &        .4    &      .2   &           -   &    -  &    -  &                -      &       -     \\
  Cappuccino  &    .4    &        .2    &      .4   &           -   &    -  &    -  &                -      &       -     \\
  Red Eye     &    .4    &         -    &       -   &          .6   &    -  &    -  &                -      &       -     \\
  Americano   &    .4    &         -    &       -   &           -   &   .6  &    -  &                -      &       -     \\
  Black Eye   &    .6    &         -    &       -   &           -   &   .4  &    -  &                -      &       -     \\
  Viennois    &    .4    &        .3    &       -   &           -   &    -  &   .3  &                -      &       -     \\
  Caffe Mocha &    .4    &        .3    &       -   &           -   &    -  &   .2  &                -      &       .1    \\
  Con Panna   &    .4    &         -    &       -   &           -   &    -  &   .4  &                -      &       -     \\
  Flat White  &    .4    &        .4    &       -   &           -   &    -  &    -  &                -      &       -     \\
  Cafe Breve  &    .4    &         -    &      .2   &           -   &    -  &    -  &               .4      &       -     \\
  Macchiatto  &    .4    &         -    &      .4   &           -   &    -  &    -  &                -      &       -     \\
\end{tabular}
\normalsize

% The most ingredients in any of the drinks is 4.

\def\scale{1.5}  % how big is the cup?
\def\espresso{.9}    % how full is the cup?
\def\cupwall{5}  % how thick is the cup wall?
\def\cuptype{6}  % how flat it the cup's bottom?

\begin{figure}
  \begin{center}
    \begin{tikzpicture}

      % keep the drink in the cup
      \clip[scale=\scale,domain=-1.1:1.1,smooth,variable=\x]
      plot ({\x},{pow(\x,\cuptype)});

      % put espresso in the cup
      \draw[scale=\scale, brown, fill=brown,decorate,
        decoration = {snake,amplitude =.1mm, segment length = 5mm} ]
      (-2,-2) rectangle (2,{\espresso * pow(1.1,\cuptype)});

      % draw the cup around our magically constrained drink
      \draw[scale=\scale,domain=-1.1:1.1,smooth,variable=\x,
        name path=cup,black,line width=\cupwall]
      plot ({\x},{pow(\x,\cuptype)});

    \end{tikzpicture}
  \end{center}
  \caption{Type \cuptype\ Cup}
\end{figure}


\end{document}
